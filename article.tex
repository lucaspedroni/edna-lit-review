\documentclass[letterpaper, 10pt, hidelinks]{article}

\usepackage{fullpage}
\usepackage{hyperref}
\usepackage{helvet}
\usepackage{booktabs}
\usepackage{color}
\usepackage{soul}
\usepackage{graphicx}

\begin{document}
\section{Introduction}

Environmental DNA (eDNA) involves the extraction and analysis of DNA from environmental samples including 
soil, marine sediment, water, and other media. eDNA techniques have created unprecedented opportunities for 
a number of ecologically important tasks like biodiversity management and invasive or endangered species 
monitoring. 
Research utilizing eDNA techniques has increased dramatically in recent years, and this work 
aims to provide relevant background information and an overview of current research trends involving 
eDNA.

\section{History}

Environmental DNA extraction and analysis techniques have been utilized for more than two 
decades~\cite{Paul1990, Trevors1989, Somerville1989}. As these techniques have become more widely 
adopted and eDNA research more publicized, researchers have directly adopted the term 
``environmental DNA'' to describe a set of common tools used to extract, amplify, sequence, and 
analyze DNA extracted from environmental media. The term became popularized after its 
increased usage in works related to biodiversity and ecology, most notably research published 
by Ficetola et al.\ in 2008~\cite{Ficetola2008}. Ficetola et al.\ used eDNA extracted from 
water samples in order to track American bullfrog presence in different environments.

\section{Technological Breakthroughs}

High throughput sequencing has allowed for increased identification resolution.

\section{Research Trends}

Ecological biodiversity and invasive species are logical applications for eDNA techniques, and 
eDNA techniques have consistently been utilized for these tasks. These techniques may be 
readily applied to numerous other tasks and have been creatively applied to 
a number of different domains. 

eDNA techniques have been widely utilized to analyze the gut microbiomes of a variety of 
organisms including humans~\cite{Turnbaugh2009}, mosquitoes~\cite{Osei-Poku2012}, and
dogs~\cite{Swanson2011}. In addition to characterizing organisms' microbiomes, eDNA 
techniques with fecal samples are being used to characterize organisms' raw diets. 
Lyke et al.\ used eDNA techniques to identify arthopod families from DNA extracted 
from redtail and blue monkey fecal samples~\cite{Lyke2019}. Similarly, horseshoe bat 
insect diets have been categorized using fecal samples~\cite{Aldasoro2019}.

\section{Challenges}

\section{Future Directions}

\bibliography{refs}{}
\bibliographystyle{plain}

\end{document}
