\documentclass[letterpaper, 10pt, hidelinks]{article}

%\usepackage{apacite}
\usepackage{fullpage}
\usepackage{hyperref}
\usepackage{helvet}
\usepackage{booktabs}
\usepackage{color}
\usepackage{soul}
\usepackage{graphicx}
\usepackage{titlesec}

\titleformat{\section}
	{\normalfont\normalsize\bfseries}
	{}
	{0pt}
	{}

\begin{document}
\section{Introduction}

Environmental DNA (eDNA) involves the extraction and analysis of DNA from environmental samples including 
soil, marine sediment, water, and other media. eDNA techniques have created unprecedented opportunities for 
a number of ecologically important tasks like biodiversity management and invasive or endangered species 
monitoring. 
Research utilizing eDNA techniques has increased dramatically in recent years, and this work 
aims to provide relevant background information and an overview of current research trends involving 
eDNA.

\section{History}

Environmental DNA extraction and analysis techniques have been utilized for more than two 
decades~\cite{Paul1990, Trevors1989, Somerville1989}. As these techniques have become more widely 
adopted and eDNA research more publicized, researchers have adopted the term 
``environmental DNA'' to describe a set of common tools used to extract, amplify, sequence, and 
analyze DNA extracted from environmental media. The term became popularized after its 
increased usage in works related to biodiversity and ecology, most notably research published 
by Ficetola et al.\ in 2008~\cite{Ficetola2008}. Ficetola et al.\ used eDNA extracted from 
water samples in order to track American bullfrog presence in different environments.

In 2003, the cytochrome c oxidase I (\emph{COI}) gene was published as a putative 
gene target for a global, sustainable barcoding identification process to identify 
animals~\cite{Hebert2003}. \emph{COI}, a mitochondrial gene, is frequently used for 
barcoding purposes in research. 

\section{Technological Breakthroughs}

High throughput sequencing has allowed for increased identification resolution.

\section{Research Trends}

eDNA techniques have been consistently utilized for ecological biodiversity monitoring 
tasks. These techniques may be 
readily applied to numerous other tasks and have been creatively applied to 
a number of different domains. 

eDNA techniques have been widely utilized to analyze the gut microbiomes of a variety of 
organisms including humans~\cite{Turnbaugh2009}, mosquitoes~\cite{Osei-Poku2012}, and
dogs~\cite{Swanson2011}. In addition to characterizing organisms' microbiomes, eDNA 
techniques with fecal samples are being used to characterize organisms' raw diets. 
Lyke et al.\ used eDNA techniques to identify arthopod families from DNA extracted 
from redtail and blue monkey fecal samples~\cite{Lyke2019}. Similarly, bat diets 
have been analyzed using fecal samples~\cite{Aldasoro2019, Razgour2011}. eDNA analysis, 
with the use of high throughput sequencing, proved particularly useful in Razgour et al.'s 
work identifying resource partitioning among cryptic (morphologically similar but 
genetically isolated) bat species~\cite{Razgour2011}. 

\section{Challenges}

The effectiveness of using eDNA for species detection and biodiversity is debated 
in published research. eDNA techniques have various limitations but are extremely 
useful for use in environments where morphological identification, either through 
observation or trapping, is difficult. The efficacy of using eDNA for species 
identification may vary drastically depending on very specific circumstances. 
eDNA methods have proved useful in 
marine environments where fish populations exist among numerous other divers species. Yamamoto 
et al.\ found that eDNA techniques outperformed visual surveys for detection of 
marine fish in species-rich coastal areas of Japan~\cite{Yamamoto2017}. However, 
eDNA analysis may have high potential for false negatives. 
Trapping survey methods were found to outperform eDNA detection of semi-aquatic 
snakes in California~\cite{Rose2019}.
Small populations of 
solitary organisms in volatile environments may be particularly hard to detect using 
eDNA. Research regarding DNA detectability in freshwater have produced varying 
results, with some work showing a sharp dropoff in detectability after a few days 
but a lower potential for detection for up to two weeks after removal of an organism 
from an environment~\cite{Dejean2011}. Work examining stream-dwelling salamander 
eDNA degradation found that less than 1\% of original eDNA concentrations remained 
three days after removal of salamanders from the site, and eDNA was not at all 
detectable for samples receiving eight days of full sun exposure~\cite{Pilliod2014}. 
eDNA has shown to be detectable in flowing waters up to almost 10 km in one 
case~\cite{Deiner2014}, but these results are clearly the result of the interactions 
of a huge number of variables. Understanding the variables affecting eDNA degradation 
and detectability are critical to the development of eDNA methods. There has been 
research attemping to quantify eDNA degradation under controlled conditions and correlate 
degradation to variables like biochemical oxygen demand, chlorophyll content, and 
total eDNA~\cite{Barnes2014}. There is still a great deal of work to be done understanding 
how environmental conditions may affect eDNA experimental outcomes. For research 
utilizing eDNA, it may be necessary to perform additional experimental work attempting 
to quantify and understand eDNA degradation parameters for each specific environment 
that samples are collected from.

\section{Future Directions}

\bibliographystyle{plain}
\bibliography{refs.bib}{}

\end{document}
