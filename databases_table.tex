\documentclass[letterpaper,10pt]{article}
\usepackage{booktabs}
\usepackage{hyperref}
\usepackage{helvet}
\usepackage{fullpage}

\begin{document}
\begin{table}[h]
	\centering
	\begin{tabular}{p{4cm} p{4cm} p{7cm}}
		\toprule
		{Resource} & {URL} & {Overview} \\
		\midrule
		{Barcode of Life Data System (BOLD)} & {\url{boldsystems.org}} & {Analysis platform and workbench for 
		DNA barcoding experiments} \\
		{International Barcode of Life (iBOL)} & {\url{ibol.org}} & {Barcode reference library project, consortium 
		devoted to monitoring biodiversity. \$180 million project to identify new species.} \\
		{Barcode of Life} & {\url{ncbi.nlm.nih.gov/genbank/barcode}} & {NCBI's barcoding reference project, has an 
		interactive web tool for submissino of barcode sequences} \\
		{OBITools} & {\url{git.metabarcoding.org/obitools/obitools/wikis/home}} & {Python programs developed for 
		NGS data in the context of metabarcoding. Includes tools for designing primers, evaluating primers, 
		alignment, and clustering} \\
		{Spider} & {\url{cran.r-project.org/web/packages/spider/index.html}} & {R package for analyzing DNA 
		barcoding data. Includes functions for summary statistics, identification efficacy evaluation, and numerous 
		other features} \\
		{Fish Barcode Information System (FBIS)} & {\url{mail.nbfgr.res.in/fbis}} & {Web tools for taxonomy, species 
		identification, and sequence diversity estimation - specifically for fishes of the Indian subcontinent} \\
		{} & {} & {} \\
		{jMOTU and Taxonerator} & {\url{nematodes.org/bioinformatics}} & {GUI software for clustering barcode DNA 
		sequence data into molecular operational taxonomic units and annotating base on similarity} \\

		\bottomrule
	\end{tabular}
\end{table}

\end{document}
